\documentclass[11pt]{article}

% Language setting
\usepackage[portuguese]{babel}

% Set page size and margins
\usepackage[a4paper,top=2cm,bottom=2cm,left=3cm,right=3cm,marginparwidth=1.75cm]{geometry}

% Useful packages
\usepackage{amsmath}
\usepackage{graphicx}
\usepackage{minted}
\usepackage{fancyhdr}
\usepackage{lastpage}
\usepackage[all]{xy}
\usepackage[colorlinks=true, allcolors=blue]{hyperref}

\title{Sistemas Operativos}
\author{Grupo 107 \\ 62608 - Marco Sousa \\ 93271 - José Malheiro}

\begin{document}
\maketitle

\begin{abstract}
    Este projeto permitiu desenvolver competências de \textbf{Engenharia de Software},
    nomeadamente \textit{gestão de processos e execução concorrente } de programas.

    O objetivo foi implementar um serviço capaz de transformar ficheiros de áudio por aplicação de uma sequência de filtros.
    Para além deste, permite ainda consultar as tarefas em execução e número de filtros disponíveis e em uso.
\end{abstract}

\section{Introdução}

O presente relatório foi redigido no âmbito da unidade curricular (UC) \underline{Sistemas Operativos}
e remete-se à elaboração de um projeto na linguagem de programação C para um \textbf{Serviço de Processamento de Áudio}.

A construção do projeto teve como referência a orientação dos docentes da UC e principal objetivo de desenvolver
conceitos de gestão de processos, criação de filhos, utilização de sinais, direcionamento de descritores \textit{STDIN, STDOUT, STDERR}.
Para além destes, permitiu o aprofundamento sobre o desenvolvimento de programas na linguagem C.

\section{Estrutura do Projeto}

A arquitetura utilizada foi uma estratégia modular e encapsulada que permitiu a criação de um sistema cliente-servidor fácil de manter e re-estruturar.

\begin{itemize}
    \item Client
          \begin{itemize}
              \item Processo que permite fazer pedidos ao servidor (cliente);
          \end{itemize}
    \item Modules
          \begin{itemize}
              \item Módulos de apoio ao desenvolvimento
              \item Pool
                    \begin{itemize}
                        \item Implementação do tipo \textit{thread-pool}
                    \end{itemize}
          \end{itemize}
    \item Server
          \begin{itemize}
              \item Processo que recebe os pedidos e encaminha para a pool
          \end{itemize}
\end{itemize}

Por forma a verificar a funcionalidade do sistema, é necessário executar o processo servidor com o comando:
\begin{minted}{bash}
    > bin/aurrasd <path_to_config> <path_to_filters>
\end{minted}

Posteriormente, pode ser executado qualquer cliente de forma concorrente, utilizando um dos seguintes comandos:
\begin{minted}{bash}
    > bin/aurras status
    > bin/aurras transform <input_file> <output_file> <filter_1 ... filter_n>
\end{minted}

Para gerar os executáveis, deverá executar-se na \textit{bash} o comando \mintinline{bash}{make}.

\begin{equation*}
    \xymatrixcolsep{5pc}\xymatrix {
    Client\ar[r]_{<request>}^{Pipe} & Server\ar[r]_{<request>}^{Pipe} & Pool\ar[r]_{get\_task} &\ar@/_2pc/[d]_{exec\_full}\ar@/^2pc/[d]^{exec\_partial}\\
    & & & termina\ar@/^3pc/[ulll]^{warns\_client}_{Pipe} \ar@/_6pc/[ul]_{add\_to\_queue}^{signals}
    }
\end{equation*}

\subsection{Estruturas de Dados}

As estruturas de dados desenvolvidas para auxiliar na gestão do sistema, foram \textit{Config}, \textit{Pool}, \textit{Queue} e \textit{Task}.

Cada uma destas estruturas Apresenta uma gestão própria, sendo-lhe disponibilizado uma API com funções para inicializar, adicionar,
remover ou atualizar dados, etc. Estão devidamente documentadas para facilitar a uilização por outros para além do seu programador
original.

\begin{equation*}
    \xymatrixcolsep{5pc}\xymatrix {
    & Pool \\
    Config\ar@/^/[ur] & & Queue\ar@/_/[ul]\ar[d]_{one\_to\_many} \\
    & & Task
    }
\end{equation*}

\subsubsection{Config} \label{config}

\begin{minted}{c}
            typedef struct config *Config;
            typedef struct config_server *Config_Server;
            
            struct config
            {
                char *filter;
                char *file_name;
                int current, max;
            } CNode;

            struct config_server
            {
                Config *config;
                char *configFolder;
                int total;
                int size;
            } CSNode;

			\end{minted}

Estrutura utilizada para armazenar e gerir a configuração do servidor, nomeadamente caminho para os filtros,
filtros disponíveis e respetivo número de instâncias em uso e máximas. Ver \ref{config_api}.

\subsubsection{Task}\label{task}

\begin{minted}{C}
        typedef struct task *Task;

        typedef enum taskstatus
        {
            PENDING,
            WAITING,
            PROCESSING,
            FINISHED,
            ERROR
        } Status;

        typedef enum execute_type
        {
            FULL,
            PARTIAL
        } ExecuteType;

        typedef struct filters
        {
            char **filters; // filtros com a ordem pedida pelo utilizador
            int size;       // size of array
            int num_filters;
            int current_filter;
            char **filters_set; // set de filtros (sem repetições)
            int *filters_count; // lista de contadores (relacionado com filtro)
            int num_filters_set;
        } * Filters, FNode;
        struct task
        {
            char *request;         // pedido completo
            char *command;         // transform vs status
            char *pid;             // pid do cliente que pediu
            int client_fd;         // fd to pipe com cliente
            ExecuteType execution; // tipo de execução (parcial ou completa)
            int executer_pid;      // pid do filho que está a executar
            Status status;         // estado do processamento
            char *input_file;      // path para file input
            char *output_file;     // path para file output
            Filters filters;
        } TNode;
    
\end{minted}

Cada pedido que o cliente efetua, corresponde a uma tarefa no servidor. Esta tarefa tem um estado próprio,
que permite fazer a sua gestão. Ver \ref{task_api}

\begin{itemize}
    \item [comunicar com cliente] {pid do cliente\\ descritor para pipe}
    \item [execução do comando] {filtros associados, por ordem \\ pid do filho a executar comando \\ tipo de execução (parcial ou completa)
          estado de processamento \\ caminho para input \\ caminho para output}
    \item [gestão da execução] { filtro em execução (execução parcial)}
    \item [API] {adicionar filtros \\ obter número de filtros \\ obter próximo filtro a executar \\ obter filtro em execução
          set de filtros (sem repetições) \\ set de contador de instâncias necessárias}
\end{itemize}

Assim, quando uma tarefa é iniciada, recebe como argumento o \textit{request} realizado pelo cliente, com a
seguinte estrutura:
\begin{minted}{bash}
    $ <pid_cliente> <transform> <input_file> <output_file> <filtros>
    $ <pid_cliente> <status>
\end{minted}

A partir desta string, é extraída toda a informação necessária através do seu parsing, conforme descrito.
É ainda verificado se o ficheiro de input efetivamente existe. Caso não exista, a tarefa não é adiciona à queue
e é apresentado uma mensagem de erro no cliente, assim como fechado o seu descritor de escrita do pipe.

Para facilitar a gestão, foi criado duas estruturas do tipo \mint{C}{enum}, nomeadamente para saber qual o estado de processamento
da tarefa e qual o tipo de execução.

Em relação a este último, salienta-se a sua importância pois o sistema desenvolvido permite dois tipos de excução: quando
o servidor tem recursos suficientes para executar de uma só vez (através de uma \textit{pipeline}) - execução completa
ou quando tal não é possível, aplica um filtro de cada vez. Neste caso, é criado um ficheiro temporário para ser utilizado
na iteração seguinte.

\subsubsection{Queue} \label{queue}

\begin{minted}{C}
            struct queue
            {
                Task *tasks;
                int current;
                int pending;
                int total;
                int size;
            } NQueue;
            typedef struct queue *Queue;
\end{minted}

O conceito inicial \textbf{Queue} seria uma estratégia do tipo FCFS (\textit{First Come First Served}).
No entanto, ao longo da realização do trabalho, tal como descrito na literatura, esta estratégia não é, de todo, eficiente.
Este tópico será discutido na secção \ref{pool}.

Através desta estrutura, é possível ter conhecimento de quantas tarefas estão pendentes, total (em precessamento e pendentes) e
tarefa atual. Com a utilização da API, ver \ref{queue_api}, é possível fazer a gestão da lista de espera através de funções
para adicionar, atualizar, remover e obter tarefas.

\subsubsection{Exec Helper}\label{exec_helper}

Com apenas duas funções, esta API permite a execução total ou parcial de uma tarefa.

É utilizada pela Pool (ver \ref{pool}) para apoio na execução de tarefas pendentes, mediante o tipo de execução associada (\textit{PARTIAL vs FULL}).

Para mais informação, ver API \ref{exec_helper_api}.

\subsubsection{Pool} \label{pool}

O módulo \textbf{Pool} serve como um controlador. Permite fazer a gestão dos processos que o cliente pede, desde a
criação de \textit{tasks}, verificação se são válidas e adição à \textit{queue} até à execução de processos de forma concorrente.

\begin{minted}{C}
    static Pool POOL;

    struct pool
    {
        Queue queue;
        Config_Server config;
    } PNode;
\end{minted}

Como já abordado, a abordagem inicial seria uma estratégia do tipo FCFS. No entanto, devido ao entusiasmo e vontade de desenvolver uma solução
mais completa, verificou-se que tal estratégia não seria compatível. Pelo menos de uma forma eficiente.

Assim, utiliza-se uma variação da estratégia FCFS, i.e. o servidor tenta executar uma tarefa assim que uma nova é recebida.
Se for a sua vez, tenta executá-la.
Caso não tenha recursos, procura a tarefa seguinte que seja possível de executar com os recursos disponíveis.
Caso não seja possível executar mais nenhuma tarefa,
volta para o \textit{read} e aguarda por mais tarefas ou que sejam libertados recursos.

\begin{equation*}
    \xymatrixcolsep{4pc}\xymatrix{
    & Pool\ar[l]_{close\_descriptor}\ar[r]^{remove\_task}\ar@(ur, ul)_{set\_waiting} & \\
    & & \\
    Read\ar[r]_{request} & init\_task\ar@/_3pc/[l]_{task\_invalida}\ar[r]_{task\_valida} & add\_to\_queue\ar[d]^{handle\_pool} \\
    & & handle\_pool\ar@/^1pc/[d]^{get\_next\_task}\ar@/^4pc/[dd] \\
    finaliza task\ar@/_3pc/[uuuur]_{signal}^{release\_resources} & multiplos\_forks+dup2+exec\ar[l] & execucao\_completa\ar[l] \\
    finaliza\_vs\_temporario\ar@/_4pc/[uuuuur] & aplica\_filtro\ar[l] & execucao\_parcial\ar[l]_{get\_next\_filter} \\
    }
\end{equation*}

Através da estratégia utilizada, permite a execução, quer de pedidos que o servidor seja capaz de resolver de uma só vez (mesmo que esperando)
pela sua disponibilidade, quer aqueles que nunca seria capaz e causaria \textit{starving} caso entrasse na queue.

De forma sucinta:
\begin{itemize}
    \item Recebe uma request
    \item Processa e cria uma \textit{task}
    \item Verifica se a task é válida (caso não seja, responde ao cliente)
    \item Adiciona a task à queue
    \item Tenta executar a próxima task na queue
    \item Possível:
          \begin{itemize}
              \item Verifica o tipo de excução (parcial ou total)
              \item Executa e ocupa os recursos necessários
              \item A execução é feita criando filhos para tal. ver \ref{exec_helper}
              \item Quando estes terminam, enviam sinal ao pai a informar o seu término
              \item O pai (pool) recebe a interrupção e verifica quem terminou
              \item Vai à \textit{queue} verificar de que task se trata
              \item Verifica se terminou a query ou ainda há mais para processar
              \item Coloca query como waiting no último caso
              \item Remove-a da \textit{queue} no primeiro
              \item Liberta os recursos que a mesma estava a utilizar
              \item Volta para o \textit{handle\_pool}
          \end{itemize}
    \item Não possível
          \begin{itemize}
              \item Volta para o modo bloqueado (\textit{read}) a aguardar novas tasks OU interrupção de processos que terminaram e vão libertar recursos
          \end{itemize}
\end{itemize}

\subsection{Client - Aurras} \label{aurras}

A forma de efetuar pedidos ao servidor é através de um \textit{Cliente} com as seguintes funcionalidades:

\begin{itemize}
    \item [Ficheiro FIFO] {PID\\
          Criado pipe com nome do PID do cliente\\
          Pipe que será utilizado para receber estado de processamento do pedido\\
          Abertura do extremo de leitura, ficando a aguardar em espera passiva}
    \item [Comunicação Server] {SERVER\_REQUEST\\
          Abertura de extremo de escrita do pipe de comunicação com servidor\\
          Utilizado para fazer o pedido}
    \item [Gestão de sinais] {\textit{SIGINT} e \textit{SIGTERM}\\
          É utilizado a função \textbf{handle\_term} por forma a garantir que o ficheiro pipe gerado, é apagado aquando
          do término da execução do cliente. Mesmo que o término seja forçado.}
    \item[Helper] {Caso a execução seja efetuada sem argumentos, imprime no ecrã o formato de pedidos aceite}
\end{itemize}

Posto isto, o cliente recebe o pedido sob a forma de argumentos (na chamada da bash), podendo ser do tipo:
\begin{minted}{bash}
    > aurras status
    > aurras transform <input_file> <output_file> <filter_1> ... <filter_n>
\end{minted}

O primeiro, irá apresentar o estado do servidor, nomeadamente quais as tarefas com estado WAITING, PENDING, PROCESSING, o número de instâncias em uso
e número máximo das mesmas e o PID do servidor (ponto de entrada dos pedidos). Formato:
\begin{minted}{bash}
$ bin/aurras status
Task #1: status  PROCESSING
path filtros: bin/aurrasd-filters
alto aurrasd-gain-double: 0/2 (current/max)
baixo aurrasd-gain-half: 0/2 (current/max)
eco aurrasd-echo: 0/2 (current/max)
rapido aurrasd-tempo-double: 0/2 (current/max)
lento aurrasd-tempo-half: 0/2 (current/max)
pid: 1729
\end{minted}

O segundo tipo, irá aplicar uma série de filtros pedidos ao input e armazenar o resultado no output:
\begin{minted}{bash}
$ bin/aurras transform samples/sample-1-so.m4a output_cenas.mp3 alto baixo
Processing 2 filters
Terminated
\end{minted}

Conforme já abordado, o sistema desenvolvido permite a aplicação de filtros mesmo que o número de filtros exceda o limite de instâncias. Tal acontece através
da criação de ficheiros temporários:
\begin{minted}{bash}
$ bin/aurras transform samples/sample-1-so.m4a output_cenas.mp3 alto baixo eco eco eco
Processing filter #1: alto
Processing filter #2: baixo
Processing filter #3: eco
Processing filter #4: eco
Processing filter #5: eco
\end{minted}

Os ficheiros temporários gerados são eliminados assim que deixam de ser necessários.
Notavelmente, é uma solução menos eficiente pois implica persistência em disco entre cada aplicação de filtro.

\subsection{Server - aurrasd} \label{aurrasd}

Efetua a inicialização do servidor, criando todos os recursos necessários para a gestão de pedidos, a saber:
\begin{itemize}
    \item[POOL\_PIPE] {Cria um pipe com nome para comunicação com a instância pool\\
          Efetuado fork()\\
          Filho executar a pool (instância que faz a gestão dos pedidos)}
    \item[comuniação cliente] {REQUEST\_SERVER\\
          Cria o ficheiro fifo que aceita os pedidos de execução dos clientes\\
          Abre, ainda, o extremo de escrita do pipe, por forma a ficar aberto durante toda a execução do servidor, garantindo uma espera passiva}
    \item[read] {Fica em espera passiva por escritas no pipe de comunicação com o cliente-servidor}
\end{itemize}

A inicialização do processo \textit{aurrasd} compreende ao seguinte comando:
\begin{minted}{bash}
    > aurrasd <config_file> <filters_folder>
\end{minted}

\begin{itemize}
    \item[config\_file]{$\langle nome\_filtro\rangle \langle ficheiro\_filtro\rangle \langle max\_instancias \rangle $ \\
          cada linha deve ter a estrutura indicada\\
          É carregado na configuração do servidor. Ver \ref{config}}
    \item[filters\_folder] {path para pasta com os filtros\\
          cada filtro tem um ficheiro associado que deve estar no caminho indicado}
\end{itemize}

Caso se tente executar o servidor sem os argumentos necessários (seja a mais ou a menos), é retornada uma mensagem de erro
e o mesmo é terminado.

\begin{minted}{bash}
$ bin/aurrasd op1
helper:
> aurrasd <config_file> <filters_folder>

$ bin/aurrasd op1 op2 op3
helper:
> aurrasd <config_file> <filters_folder>
\end{minted}

\section{Considerações Finais}

Através da construção do \textbf{Serviço de Processamento de Áudio}, foi possível fortalecer conceitos desenvolvidos
ao longo da unidade curricular (UC) \textbf{Sistemas Operativos}.
Não obstante, foi também importante todo o conhecimento adquirido ao longo do curso de \textbf{Engenharia Informática},
nomeadamente estruturas de dados e modularidade.

A principal dificuldade durante a realização do \textbf{Serviço} foi a arquitetura da aplicação, definir a forma de comunicação
entre cliente e servidor e gestão dos processos em execução, pendentes, em espera.
Depois de se conseguir definir a estratégia, a sua implementação foi relativamente linear, sem grandes dificuldades.

No entanto, a aplicação desenvolvida apresenta algumas limitações, sejam elas:
\begin{itemize}
    \item[sinais concorrentes] {limite da fila de espera de sinais concorrentes\\
          pode levar a comportamento errático \\
          não aconteceu durante os testes executados}
    \item[máximo de tarefas] {o máximo de tarefas na queue, encontra-se limitada a 100\\
          no entanto, esta limitação é facilmente ultrapassável\\
          bastaria implementar uma queue com tamanho dinâmico, em vez de estático }
    \item[término 'elegante'] {garantir que, quando o servidor recebe um sinal do tipo \textit{SIGTERM} permite as
          tarefas pendentes terminarem, não aceitando mais\\
          Foi implementada uma solução que não foi possível fazer debug, apesar de, conceptualmente, ser funcional}
\end{itemize}

Apesar destas limitações, a solução desenvolvida encontra-se funcional e utilizou-se os conceitos alvo da UC em estudo, seja:
\begin{itemize}
    \item Fila de espera
    \item Utilização de esperas passivas
    \item Criação de filhos
    \item Direcionamento de descritores
    \item Gestão de processos
    \item Processos concorrentes
    \item \dots
\end{itemize}

Desta forma, acredita-se que se atingiu os objetivos da UC de forma satisfatória.

\section{Anexos}

\subsection{Config API} \label{config_api}

\inputminted{C}{../include/modules/config.h}

\subsection{Task API} \label{task_api}

\inputminted{C}{../include/modules/task.h}

\subsection{Queue API} \label{queue_api}

\inputminted{C}{../include/modules/queue.h}

\subsection{Exec Helper API} \label{exec_helper_api}

\inputminted{C}{../include/modules/exec_helper.h}

\subsection{Pool API} \label{pool_api}

\inputminted{C}{../include/modules/pool.h}

\subsection{Client API} \label{client_api}

\inputminted{C}{../src/client/aurras.c}

\end{document}